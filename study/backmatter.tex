\documentclass[10pt,aspectratio=169,dvipsnames]{beamer} % sets document type, default font size, slide aspect ratio, and loads color names
\usetheme[color/block=transparent]{metropolis} % sets the theme of the document

\usepackage[absolute,overlay]{textpos} % allows absolute positioning of text
\usepackage{booktabs} % enhances quality of tables
\usepackage[utf8]{inputenc} % allows input encoding in UTF-8
\usepackage{tikz} % used for creating vector graphics
\usetikzlibrary{arrows.meta} % loads additional arrow types
\usepackage[europeanresistors,americaninductors]{circuitikz} % for drawing electrical circuits
\usepackage[scale=2]{ccicons} % loads Creative Commons icons
\usepackage[official]{eurosym} % loads the official symbol for the Euro
\usepackage{hyperref} % allows creating hyperlinks in the document
\usepackage{fontawesome}

\newcommand{\ra}[1]{\renewcommand{\arraystretch}{#1}} % creates command to adjust spacing between rows
\newcommand{\hrefc}[2]{\href{#1}{\bf\color{blue}{\underline{#2}}}} % defines command for underlined, blue hyperlink
\newcommand{\urlc}[1]{\hrefc{#1}{#1}} % defines command for URL hyperlink

\newcommand{\R}{\mathbb{R}} % creates a shortcut for typing real numbers symbol
\newcommand{\ubar}[1]{\text{\b{$#1$}}} % defines a command for underlined text

\xdefinecolor{TUred}{RGB}{197,14,31} % defines a new color TUred
\setbeamerfont{alerted text}{series=\bfseries} % sets the font of alerted text to bold
\setbeamercolor{alerted text}{fg=TUred} % sets the color of alerted text to TUred
\setbeamercolor{background canvas}{bg=white} % sets the background color to white
\setbeamercolor{frametitle}{bg=lightgray!40, fg=TUred} % sets the background color of the frame title to light gray and text color to TUred
\setbeamercolor{title}{fg=TUred} % sets the color of the title to TUred

\addtobeamertemplate{frametitle}{}{% adds image to every frame title
  \begin{textblock*}{100mm}(1.01\textwidth,2pt)
    \includegraphics[width=1.5cm]{images/TUB.png}
    \end{textblock*}}

\def\l{\lambda} % defines a shortcut for lambda symbol
\def\m{\mu} % defines a shortcut for mu symbol
\def\d{\partial} % defines a shortcut for partial symbol
\def\cL{\mathcal{L}} % defines a shortcut for caligraphic L symbol
\def\co{CO${}_2$} % defines a shortcut for CO2 symbol
\def\el{${}_{el}$} % defines a subscript for el
\def\th{${}_{th}$} % defines a subscript for th
\def\gas{${}_{gas}$} % defines a subscript for gas

\setbeamercolor{framesource}{fg=gray} % sets color of framesource to gray
\setbeamerfont{framesource}{size=\tiny} % sets font size of framesource to tiny
\newcommand{\source}[1]{% creates command for inserting a source footnote
\begin{textblock*}{5cm}(10.5cm,8.35cm)
    \begin{beamercolorbox}[ht=0.5cm,right]{framesource}
        \usebeamerfont{framesource}\usebeamercolor[fg]{framesource} {#1}
    \end{beamercolorbox}
\end{textblock*}}

\graphicspath{{../results/}} % sets the path where graphics can be found
\DeclareGraphicsExtensions{.pdf,.jpeg,.png,.jpg} % defines the types of graphic files that can be used

\def\goat#1{{\scriptsize\color{green}{[#1]}}} % defines a command for green, scriptsize text

\let\olditem\item % saves the old item command
\renewcommand{\item}{\olditem\vspace{5pt}} % redefines the item command to add space after each item
